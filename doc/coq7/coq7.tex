\documentclass{article}

\usepackage{wrapfig}
\usepackage[colorlinks=true]{hyperref}
\usepackage{graphicx}
\usepackage{listings}
\usepackage[utf8]{inputenc}
\usepackage{verbatim}

\author{
	Alexander Faithfull\\IT University of Copenhagen\\\texttt{alef@itu.dk}
	\and
	Jesper Bengtson\\IT University of Copenhagen\\\texttt{jebe@itu.dk}}
\title{Coqoon: further adventures in\\Coq IDE development}

\date{}

\begin{document}

\maketitle

\noindent
At last year's Coq Workshop in Vienna, we presented Coqoon\cite{Faithfull2014Coqoon},
an IDE for Coq with sophisticated features like syntax highlighting,
context-aware search functionality driven by a rich model of Coq code,
and an integrated build system with automatic dependency
resolution.

We continue to work on Coqoon, and we discuss some recent developments below.

\begin{comment}

Since then, we have added (experimental) support for the PIDE protocol,
bringing automatic and asynchronous re-evaluation of proofs; added a user
interface for managing Coq load paths; and implemented a
clone of Coqoon's build system in Python, making it even easier for people to
work on Coqoon projects without having to use it.

\end{comment}

\subsubsection*{Asynchronous proofs with PIDE}

In another of the presentations at last year's Coq Workshop, Tankink presented
a prototype Coq editor\cite{Tankink2014PIDE} using the PIDE protocol to
communicate asynchronously with the Coq backend. As part of this work, PIDE,
originally
developed as part of the Isabelle proof assistant, was ported to Coq, built
atop the
STM mechanism added in Coq 8.5.

Working with Tankink in the months after the Coq workshop, we have added
support for the PIDE protocol to Coqoon. The PIDE editor, the default since
version 0.5.0, automatically re-evaluates commands and their dependencies
whenever they change.

\paragraph{Future work}

There are still many features in the PIDE protocol that Coqoon does not yet
make use of; in particular, although PIDE sends back a parse tree for the
commands that it has evaluated, we do not yet incorporate this into the Coqoon
model of Coq code. We expect to start taking advantage of these features once
Coq 8.5 is properly released.

\subsubsection*{Load path management}

Coqoon uses the Eclipse project abstraction to give structure to Coq
developments, and its internal representation of a project's Coq load path is
richer than just a list of directories. This representation makes it possible
to portably depend on things that might not be in the same place on every
computer, like other projects in the Eclipse workspace.

Unfortunately, when
we last presented Coqoon, most of
these complex dependencies were not possible to create without manually editing
configuration files.
Version 0.4.7, released in January of this year, fixed this by adding a proper
user interface for
configuring projects' load paths.

\subsubsection*{Better support for non-Coqoon users}

To make it possible for non-Coqoon users to work with Coqoon projects, the
Coqoon build system used to finish building a project by writing out a snapshot
of its dependency tree as a Makefile with rules that simulated Coqoon's own
behaviour. However, these Makefiles only worked
reliably for simple projects---when projects depended on other projects, or on
external developments, the resulting Makefile was totally unportable. Worse
yet, the build system's constant attempts to regenerate this file often led to
version control merge conflicts.

Recent versions of Coqoon adopt a different approach. The builder, instead of
producing a project-specific Makefile, now adds a \emph{generic} configure
script to
every project; when run, this script parses the Coqoon project configuration
and produces a machine-specific Makefile from it, prompting the user if they
need to provide a path for a dependency that cannot be resolved automatically.

The generic build script changes much less frequently than the old Makefiles,
and---to minimise merge conflicts even further---Coqoon will only attempt to
update it if that particular installation of Coqoon created it in the first
place (or if the user explicitly chooses to update it).

\subsubsection*{Conclusion}

Coqoon continues to develop, and the new support for asynchronous proof editing
makes it behave more like an IDE than ever before.

More information about Coqoon, including installation instructions and links to
both precompiled Eclipse plugins and the source code, is available from
\linebreak
\url{https://coqoon.github.io/}.

\bibliographystyle{plainurl}
\bibliography{coq7}

\end{document}
