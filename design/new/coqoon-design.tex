\documentclass{article}
\usepackage[colorlinks=true]{hyperref}
\usepackage{listings}
\usepackage[utf8]{inputenc}
\usepackage{graphicx}

\author{Alexander Faithfull\\IT University of Copenhagen\\\texttt{alef@itu.dk}}
\title{Design of the 0.4 series\\of Coqoon and Kopitiam}

\newcommand{\fdef}[1]{\textit{#1}}
\newcommand{\name}[1]{\textbf{#1}}

\begin{document}
\maketitle

\abstract{Coqoon is a proof development environment for Eclipse built atop the
Coq interactive proof assistant, and Kopitiam is a proof environment for Java
programs based upon Coqoon. This paper describes the motivation behind these
projects, their evolution, architecture, and implementation, and}

\tableofcontents

\pagebreak

\section{Introduction}

\subsection{The problem}

\subsection{The goal}

\subsection{History}

\subsubsection{In the beginning}

The project began in March 2011 as \name{Kopitiam}, 

\subsubsection{Early integration}

Kopitiam 0.1, released in September 2012, integrated generated Java proof
scripts with the Java editor: the user could now step through Java statements
as though performing a normal Coq proof. This was made possible by AspectJ, a
Java implementation of the \fdef{aspect-oriented programming} methodology.

The \emph{goal} of aspect-oriented programming is to separate unrelated
concerns. An aspect-oriented logger for a system, for example, would be written
separately from that system and would be patched into it at runtime, removing
the need to clutter the original code with irrelevant logging statements.

In practice, however, the usefulness of an aspect-oriented programming system
lies in its ability to patch arbitrary code at runtime. To this day, Kopitiam
continues to use this power to extend the Java editor with Coq-like behaviour.

\subsubsection{Further integration}

Kopitiam 0.2, introduced in January 2013, ported the Java proof environment to
use Eclipse's own Java parser, removing the old SimpleJava parser.

\subsubsection{Moving beyond simple interaction}

The 0.3 series of Kopitiam, first released in June 2013, dropped support for
versions of Coq prior to 8.4, and introduced Coq projects with automatic
builder support.

\subsubsection{Consolidation and continuing integration}

The current release series, the 0.4 series, began in November 2013. This series
separated the monolithic Kopitiam project into several separate components. The
core and the Coq development environment became known as \name{Coqoon}, and
the Kopitiam name is now used exclusively for the Java proof environment.

\pagebreak

\section{Architecture}

\subsection{Plug-ins and Eclipse}

The heart of the Eclipse platform has nothing to do with IDEs at all---it's an
implementation (called \name{Equinox}) of the OSGi component model. This
component model expands upon the Java library system by adding \fdef{bundles},
packages of code which can be dynamically loaded and unloaded and whose
dependencies can automatically be resolved at runtime. The information needed
to resolve these complex dependencies is given in the bundle's \fdef{manifest}.

\begin{figure}[h]
\begin{lstlisting}[basicstyle=\footnotesize\ttfamily]
Manifest-Version: 1.0
Bundle-ManifestVersion: 2
Bundle-Name: Coqoon core
Bundle-SymbolicName: dk.itu.coqoon.core;singleton:=true
Bundle-Version: 0.4.3
Bundle-Activator: dk.itu.coqoon.core.Activator
Bundle-Vendor: IT University of Copenhagen
Require-Bundle: org.eclipse.ui,
 org.eclipse.core.runtime,
...
\end{lstlisting}
\caption{A portion of the manifest for the Coqoon core.}
\end{figure}

A bundle designed for Eclipse is known as a \fdef{plug-in}. Plug-ins can also
include additional metadata above and beyond that of the OSGi manifest: this
metadata is stored in a second XML manifest containing \fdef{extension} and
\fdef{extension point} definitions.

\subsubsection{Extensions and extension points}

Extension points define opportunities for plug-ins to extend each other. For
example, the Eclipse core defines an extension point called
\texttt{contentTypes}, which can be used by dependent plug-ins to add support
for new file types to the Eclipse platform.

Simply being able to contribute data to other plug-ins is not, by itself, very
useful, so the extension point mechanism also allows extensions to contribute
code: fully-qualified class names given as extension point attributes can be
instantiated, providing new functionality to existing code.

An extension is just a contribution to an extension point. As an example,
Coqoon has an extension for the \texttt{contentTypes} extension point which
adds support for Coq proof scripts.

\subsubsection{Lazy activation and extensions}

Making the Equinox foundation into a useful development platform requires many
hundreds of plug-ins, each of which may contain thousands of Java
classes\footnote{Coqoon and Kopitiam together have about a thousand classes;
the Eclipse JDT has over four thousand for the UI alone!}; trying to load all
of these into the JVM at once is likely to be slow---or, in
resource-constrained environments, impossible.

Eclipse addresses this problem by using \fdef{lazy activation}: that is,
a plugin is normally only activated when another activated plugin refers to it
(either directly in its code or indirectly through an extension point). But
this lazy approach can cause problems. For example, Coqoon contributes a
toolbar icon, \includegraphics[scale=0.5]{down.png}, which should only be
enabled when a Coq proof is in progress---but how can such checks be performed
without needing to load code from, and thus to activate, the relevant plugin?

\pagebreak

\section{Implementation}

Coqoon is implemented almost entirely in the Scala language, which introduces
functional programming techniques into the Java programming environment in a
way that's compatible with existing Java code

\subsection{The \texttt{core} plug-in}

The heart of Coqoon is the \texttt{dk.itu.coqoon.core} plug-in. This plugin
implements a model for Coq proof scripts, the protocol for communicating with
\texttt{coqtop} processes, and an automatic builder for Coq projects.

The \texttt{core} plug-in is also more permissively licensed than the rest of
Coqoon and Kopitiam.

\subsubsection{The \texttt{-ideslave} protocol}

Prior to Coq 8.4, the \texttt{coqtop} program had only very primitive support
for non-interactive use: the \texttt{-emacs} flag caused some extra status
information to be included in the output, but parsing that. Coq 8.4 improved this situation
dramatically by introducing the \texttt{-ideslave} protocol.

The \texttt{-ideslave} protocol is a XML-based protocol for sending requests
to Coq. Unlike the older \texttt{-emacs} mode, this protocol 

\subsubsection{Portability challenges}

The Coqoon core exposes as much Coq functionality as possible on as many
platforms as possible. This can sometimes conflict---badly---with the Java
philosophy of ``write once, run anywhere''.

As an example, the \texttt{coqtop} program allows the user to cancel a
long-running computation by sending an \fdef{interrupt signal} to the process.
Sending such a signal is impossible without the identifier of that child
process, however, and---as there is no portable way of representing a process
identifier---Java does not expose that identifier through its \texttt{Process}
API. On UNIX-like operating systems, Coqoon works around this by using a shell
script wrapper to communicate the process identifier to the Java parent, but
this is not possble on Windows, where Java runtime introspection is instead
used to read the values of private variables from the implementation of
\texttt{Process}.

It gets worse, however. On UNIX-like operating systems, signals are a core part
of the process model, and can be sent by any process to any other process; on
Windows, on the other hand, they are emulated, and can only be sent to entire
groups of processes. If a JVM running on Windows attempts to send an interrupt
signal to one of its children, it will also signal \emph{itself}---terminating
itself as a side effect!

The only way around this is to link native code into the JVM; this native code
exposes the Windows process group control primitives to the Java world, 

\subsection{The \texttt{ui} plugin}

Most of Coqoon's user interface is implemented in the
\texttt{dk.itu.coqoon.ui} plug-in.

\subsection{The \texttt{kopitiam} plugin}

The \texttt{dk.itu.sdg.kopitiam} plugin

\end{document}
